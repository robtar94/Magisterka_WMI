\section*{Wstęp}

\noindent Sztuczne sieci neuronowe są szeroko dyskutowane w literaturze naukowej. Podejmowali go różni badacze ludzkiej inteligencji, chcąc na różne sposoby zamodelować w maszynie jakiś aspekt ogólnej inteligencji i/lub działalności twórczej człowieka. Praca ta ma za zadanie poszerzyć granice naszego poznania w dziedzinie sztucznej inteligencji, podejmując temat wykorzystania metod uczenia uczenia maszynowego do realizacji zadania rozpoznawania i tworzenia muzyki.

Praca podzielona jest na sześć rozdziałów. pierwszy opisuje krótko historie i stan badań w dziedzinie sieci neuronowych. Rozdział drugi i trzeci wprowadzają niezbędny aparat pojęciowy wykorzystywany w dalszej części rozważań. Rozdziały: czwarty, piąty oraz szósty prezentują autorski system rozpoznawania i generowania muzyki, który poddany analizie skłania do pewnych przemyśleń na temat możliwości jakie oferuje sztuczna inteligencja, które sformułowane zostały w zakończeniu.


