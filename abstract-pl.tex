\noindent Sztuczne sieci neuronowe są szeroko dyskutowanym tematem w literaturze naukowej. Rozwój badań w tej dziedzinie otworzył nowe możliwości rozwiązywania problemów obliczeniowych. Za pomocą sieci neuronowych można zamodelować w maszynie działania, które do tej pory były możliwe do wykonania tylko przez człowieka. 
Niniejsza praca jest pracą interdyscyplinarną łączącą teorie muzyki z informatyką. Podejmuje w niej temat rozumienia muzyki przez maszynę. Proponowany przeze mnie autorski system, bazujący na sposobie działania typowych modeli sieci neuronowych, jest zdolny zarówno klasyfikować utwory muzyczne oraz generować nowe na podstawie istniejącego wzorca.