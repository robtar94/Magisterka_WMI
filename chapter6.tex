\chapter{Obroty wektorów nut w przestrzeni trójwymiarowej}

Wspomniane w poprzednim rozdziale macierze rotacji można wykorzystać do tworzenia muzyki.
W zdefiniowanej w rozdziale pierwszym przestrzeni wektorowej nad ciałem $\mathbb{R}$ możemy rozważyć istnienie  wektorów wartości dźwiękowych (nut) oraz operacje na nich.
Obrót każdego z takich wektorów $v$ gdzie $v \in V$ nad ciałem $\mathbb{R}$ przeprowadzimy za pomocą mnożenia przez macierze obrotu dla osi: $OX$, $OY$ oraz $OZ$. Obroty wektorów zostały zaimplementowane na dwa sposoby:
\begin{enumerate}
    \item Obroty akordów o kąt $5^{\circ}$ w przestrzeni $\mathbb{R}^{3}$,
    \item Obroty wektora nut w przestrzeni $\mathbb{R}^{3}$, reprezentującego własnoręcznie skomponowany utwór w tonacji C-dur, kolejno o kąty $5^{\circ}$, $10^{\circ}$, $15^{\circ}$ oraz $20^{\circ}$ stopni.
\end{enumerate}

\section{Obroty akordów w \texorpdfstring{$\mathbb{R}^{3}$}{R3}}
Na potrzeby eksperymentu przygotowano pogrupowane parami wektory od $v_{1}$ do $v_{4}$ następującej postaci:
\begin{equation*}
v_{1} =
    \begin{bmatrix}
     60 \\
     64 \\
     67
    \end{bmatrix}
    \quad
    v_{2} =
    \begin{bmatrix}
     69 \\
     61 \\
     64
    \end{bmatrix}
    \quad
    v_{3} =
    \begin{bmatrix}
     67 \\
     71 \\
     62
    \end{bmatrix}
    \quad
    v_{4} =
    \begin{bmatrix}
     65 \\
     69 \\
     60
    \end{bmatrix}
\end{equation*}
Następnie każdy z tych wektorów przemnożono przez macierze obrotu odpowiadające poszczególnym osiom $OX$, $OY$ oraz $OZ$ przestrzeni $\mathbb{R}^{3}$:

\begin{equation*}
R_{x}(\alpha) =
    \begin{bmatrix}
    1 & 0 & 0 \\
    0 & \cos\alpha & -\sin\alpha \\
    0 & \sin\alpha & \cos\alpha 
    \end{bmatrix}
    \end{equation*}
    \\
    \begin{equation*}
    R_{y}(\beta) =
    \begin{bmatrix}
    \cos\beta & 0 & \sin\beta \\
    0 & 1 & 0  \\
    -\sin\beta & 0 & \cos\beta
    \end{bmatrix}
    \end{equation*}
    \\
    \begin{equation*}
        R_{z}(\gamma) =
        \begin{bmatrix}
        \cos\gamma & -\sin\gamma & 0 \\
        \sin\gamma & \cos\gamma  & 0 \\
        0 & 0 & 1
        \end{bmatrix}
    \end{equation*}
    Macierz całego obrotu w $\mathbb{R}^{3}$ wyznacza się poprzez pomnożenie wektora $v$ kolejno przez macierze poszczególnych osi, tj.
    \begin{equation*}
        R = R_{x}(\alpha) \cdot R_{y}(\beta) \cdot R_{z}(\gamma).
    \end{equation*}
gdzie kąt $\alpha$ odpowiada rotacji wokół osi $OX$, kąt $\beta$ odpowiada rotacji wokół osi $OY$, a kąt $\gamma$ odpowiada rotacji wokół osi $OZ$. W niniejszym eksperymencie wszystkie wektory obrócono o tę samą wartość, czyli kąt $5^{\circ}$.

\section{Implementacja}
\begin{lstlisting}[language = Haskell]
type Vector = [Double]
type Matrix = [[Double]]

-- macierz obrotu
mx p = [[1,0,0], [0, cos p, sin p], [0, -sin p, cos p]]
my p = [[cos p, 0, - sin p], [0, 1, 0], [sin p, 0, cos p]]
mz p = [[cos p, sin p, 0], [-sin p, cos p, 0], [0, 0, 1]]
-- Dane
-- v1 do v4 to akordy
v1 :: Vector
v1 = [60,64,67]
v2 :: Vector
v2 = [69,61,64]
v3 :: Vector
v3 = [67,71,62]
v4 :: Vector
v4 = [65,69,60]
\end{lstlisting}
Wyznaczenie macierzy całego obrotu w $\mathbb{R}^{3}$:
\begin{lstlisting}[language = Haskell]
-- macierze obrotu 3D
mXY = mmult (mx 5) (my 5) 
mXYZ = mmult (mXY) (mz 5) -- cala macierz obrotu o kat 5
\end{lstlisting}

Funkcja \textit{mmult} służy do mnożenia macierzowego i została zaimplementowana następująco:
\begin{lstlisting}[language = Haskell]
mmult :: Num a => [[a]] -> [[a]] -> [[a]]
mmult a b  = [[sum $ zipWith (*) ar bc | bc <- (L.transpose b)] | ar <- a]
\end{lstlisting}
procedurę obracania wektorów zaimplementowano w poniższy sposób:
\begin{lstlisting}[language = Haskell]
wektor1 = do
    toInt(concat(mmult (w1) mXYZ))
    let nw1 = [80,37,66]
    map pitch nw1
\end{lstlisting}
funkcja \textit{pitch} służy konwersji wartości liczbowych na wartości dźwiękowe.
Następnie do każdego z wektorów dodano ćwierćnuty (qn) za pomocą następującego fragmentu kodu:
\begin{lstlisting}[language = Haskell]
--obrocony wektor 1
rw1 :: [(PitchClass, Octave)]
rw1 = [(Gs,5),(Cs,2),(Fs,4),(G,5),(G,1), (Cs, 5)]
rw1_play :: Music Pitch
rw1_play =  chord $ toMusicPitch qn (rw1)
\end{lstlisting}
Na koniec zaimplementowano monadę umożliwiającą odgrywanie parami oryginalnego akordu wraz z akordem obróconym:

\begin{lstlisting}[language = Haskell]
pierwsza_para :: Music Pitch
pierwsza_para = akord1_music :+: rw1_play

druga_para :: Music Pitch 
druga_para = akord2_music :+: rw2_play

trzecia_para :: Music Pitch 
trzecia_para = akord3_music :+: rw3_play

czwarta_para :: Music Pitch
czwarta_para = akord4_music :+: rw4_play

-- granie wszystkich par
test :: IO ()
test = do
    putStrLn "pierwsza para"
    play pierwsza_para
    putStrLn "Druga para"
    play druga_para
    putStrLn "Trzecia Para"
    play trzecia_para
    putStrLn "Czwarta para"
    play czwarta_para
\end{lstlisting}
    
\section{Wyniki}
Przez $v$ oznaczamy oryginalny wektor, a przez $v'$ wektor obrócony o kąt $\phi = 5^{\circ}$ w $\mathbb{R}^{3}$:

\begin{equation*}
v_{1} =
    \begin{bmatrix}
     60 \\
     64 \\
     67
    \end{bmatrix}
    \quad
    v'_{1} =
    \begin{bmatrix}
     80 \\
     37 \\
     66
    \end{bmatrix}
\end{equation*}
\quad
\begin{equation*}
    v_{2} =
    \begin{bmatrix}
     69 \\
     61 \\
     64
    \end{bmatrix}
    \quad
    v'_{2} =
    \begin{bmatrix}
     92 \\
     34 \\
     55
    \end{bmatrix}
\end{equation*}
\quad
\begin{equation*}
    v_{3} =
    \begin{bmatrix}
     67 \\
     71 \\
     62
    \end{bmatrix}
    \quad
    v'_{3} =
    \begin{bmatrix}
     95 \\
     42 \\
     50
    \end{bmatrix}
\end{equation*}
\quad
\begin{equation*}
    v_{4} =
    \begin{bmatrix}
     65 \\
     69 \\
     60
    \end{bmatrix}
    \quad
    v'_{4} =
    \begin{bmatrix}
     93 \\
     41 \\
     48
    \end{bmatrix}
\end{equation*}

\addtocontents{toc}{\protect\enlargethispage{\baselineskip}}
\section{Wnioski}
Wartości każdego z obróconych wektorów od $v_{1}$ do $v_{4}$ zawierają podobne wartości liczbowe co wektory oryginalnych akordów. Wynika z tego lepsza harmoniczność uzyskanych dźwięków  i ich większe podobieństwo do oryginalnych akordów. Taki efekt uzyskaliśmy dzięki obracaniu wektorów o małe wartości kąta $\phi$ w przypadku większych kątów np. kąta $90^{\circ}$ uzyskane dźwięki znacznie różniłyby się od oryginalnych, a przejście pomiędzy oryginałem, a nowo uzyskanym wektorem obróconym byłoby bardziej zauważalne, co odbiłoby się negatywnie na odczuciach estetycznych odbiorcy muzyki i harmonii samego dźwięku.

\addtocontents{toc}{\protect\enlargethispage{\baselineskip}}

\section{Obrót utworu muzycznego w \texorpdfstring{$\mathbb{R}^{3}$}{R3}}

Jako podstawę utworu bazowego zdefiniowanego w Haskellu następujące tablice dźwięków. Każda z poniższych tablic zawiera 12 dźwięków:
\begin{lstlisting}[language = Haskell]
linia1 = [C, E, G, B, D, F, A, C, E, G, B, C] 
linia2 = [C, F, A, D, G, C, F, B, E, A, D, G] 
linia3 = [C, A, F, D, B, G, E, C, A, F, D, B]
linia4 = [C, G, E, B, G, D, B, F, D, A, F, C] 
\end{lstlisting}
Następnie do każdej z tablic dołączono nuty. Funkcja \textit{pcToQN} dołącza ćwierćnuty, natomiast funkcja \textit{pcToHN} dołącza półnuty:
\begin{lstlisting}[language = Haskell]
part1 = map pcToHN (linia1)
part2 =  map pcToQN (linia2)
part3 = map pcToHN (linia3)
part4 =  map pcToQN(linia4)
\end{lstlisting}
Ostatni krok to skonwertowanie takiego zapisu w formacie \textit{dźwięk -- długość} na typ muzyczny \textit{Music Pitch}, co umożliwi odgrywanie utworu. Umożliwia to konstruktor \textit{line \$}:
\begin{lstlisting}[language = Haskell]
part1_music = line $ part1
part2_music = line $ part2
part3_music = line $ part3
part4_music = line $ part4
\end{lstlisting}
Mając skomponowany utwór, przystępujemy do jego obrotu w przestrzeni $\mathbb{R}^{3}$. W tym celu każda z linii melodycznych skonwertowaliśmy na wektory:
\begin{lstlisting}[language = Haskell]
v1 :: Vector
v1 = [0,4,7,11,2,5,9,0,4,7,11,0]
v2 :: Vector
v2 =[0,5,9,2,7,0,5,11,4,9,2,7]
v3 :: Vector
v3 = [0,9,5,2,11,7,4,0,9,5,2,11]
v4 :: Vector
v4 =[0,7,4,11,7,2,11,5,2,9,5,0]
\end{lstlisting}
Dzięki temu można wykonać operacje przemnożenia poprzez macierze obrotu osi $OX$, $OY$ i $OZ$, analogiczne do przedstawionych wcześniej.
Obrót zaimplementowano następująco:
\begin{lstlisting}[language = Haskell]
--obracanie wektorow
wektor1 = do
    toInt(concat(mmult (w1) mXYZ))
    

wektor2 = do
    toInt(concat(mmult (w2) mXYZ))
    
wektor3 = do
    toInt(concat(mmult (w3) mXYZ))
    

wektor4 = do
    toInt(concat(mmult (w4) mXYZ))
\end{lstlisting}

Każdy pojedynczy obrót umożliwiał uzyskanie nowego wektora trójwymiarowego. Dla wektorów $v_{1}$, $v_{2}$, $v_{3}$ i $v_{4}$ przy przykładowym obrocie o kąt $\phi = 20^{\circ}$ uzyskaliśmy kolejno:
\begin{equation*}
v'_{1} =
\begin{bmatrix}
7 \\
3 \\
3
\end{bmatrix}
\quad
v'_{2} =
\begin{bmatrix}
9 \\
4 \\
3
\end{bmatrix}
\quad
v'_{3} =
\begin{bmatrix}
5 \\
8 \\
4
\end{bmatrix}
\quad
v'_{4} =
\begin{bmatrix}
4 \\
6 \\
3
\end{bmatrix}
\end{equation*}
W tej procedurze testowej każdy kolejny obrót wykonywaliśmy o inny kąt, chcąc sprawdzić, czy także tym razem uzyskany utwór wynikowy będzie podobny do oryginalnego.
Składając wszystkie próby razem uzyskaliśmy utwór o długości 12 dźwięków, dokładnie tak jak w utworze oryginalnym:
\begin{lstlisting}[language = Haskell]
-- 1. Obrot o 5 stopni
rotate1 = [8,1,1,10,1,1,9,5,2,7,3,2] --linia1
-- 2. o 10 stopni
rotate2 = [3,3,7,4,4,9,7,1,8,5,1,6] -- linia2
-- 3. o 15 stopni
rotate3 = [6,5,2,8,6,3,6,9,2,4,7,1] -- linia3
-- 4. o 20 stopni 
rotate4 = [7,3,3,9,4,3,5,8,4,4,6,3] --linia4
\end{lstlisting}

Następująca monada umożliwia odgrywanie utworu oryginalnego razem z obróconym:
\begin{lstlisting}[language = Haskell]
test :: IO ()
test = do
    putStrLn "pierwszy utwor" 
    play utwor
    putStrLn "Ten sam utwor, ale po obrocie"
    play rotate_utwor
\end{lstlisting}

\addtocontents{toc}{\protect\enlargethispage{\baselineskip}}

\section{Wnioski}
Nowo uzyskany utwór zachował harmonie oryginalnego utworu, a uzyskane dźwięki wydają się podobne do oryginału. Wynika z tego następujący wniosek: Jeżeli obracamy utwór w przestrzeni nad ciałem $\mathbb{R}^{3}$, to podobnie jak w przypadku akordów należy to uczynić o małe wartości kątów, gdyż zwiększa to prawdopodobieństwo tego, że uzyskane dźwięki będą brzmieniowo zbliżone do oryginału. Nie musimy jednak  obracać utworu cały czas o ten sam kąt -- mogą to być inne wartości kątów o ile ich wartości nie są zbyt duże.