\chapter*{Zakończenie}
\noindent Celem pracy było pokazanie jak duży potencjał tkwi w metodach uczeniach maszynowego takich jak sieci neuronowe. Okazuje się, że tak klasyczny problem algorytmiczny jak klasyfikacja obiektu do jednej z predefiniowanej klasy może umożliwić rozwój dziedzin zupełnie niezwiązanych z informatyką, takich jak tworzenie muzyki -- będącej jak dotychczas niepodzielnym królestwem człowieka. Maszyna, jeżeli dobrze się ją zaprogramuje, potrafi rozpoznawać utwory muzyczne równie dobrze jak człowiek. 
Jednak potencjał metod numerycznych nie ogranicza się jedynie do zadań rozpoznawania. Maszyna jest zdolna także do komponowania muzyki. Metoda, którą tutaj pokazaliśmy ma wiele ograniczeń, ale na pewno istnieje w niej pewien potencjał, który jest warty poznania i dalszych rozważań, do których niniejsza praca miała zachęcić.