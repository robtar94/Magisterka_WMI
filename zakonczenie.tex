\chapter*{Zakończenie}
\noindent Celem pracy było pokazanie jak duży potencjał tkwi w metodach uczeniach maszynowego takich jak sieci neuronowe. Okazuje się, że tak klasyczne problemy informatyki i matematyki jak klasyfikacja obiektu do jednej z predefiniowanych klas, czy też przekształcenia geometryczne w przestrzeni, mogą stymulować rozwój dziedzin zupełnie niezwiązanych z informatyką, takich jak tworzenie i rozumienie muzyki -- będących jak dotąd możliwymi do uprawiania wyłącznie przez człowieka. Maszyna, jeżeli dobrze się ją zaprogramuje, potrafi rozpoznawać i tworzyć utwory muzyczne równie dobrze jak człowiek. Użycie nawet elementarnych właściwości sieci neuronowych daje satysfakcjonujące efekty badawcze, które wraz z ich stopniowym rozwojem zwiększają możliwości obliczeniowego modelowania różnych zjawisk. Budowa nawet prostej sieci neuronowej od podstaw nie jest sprawą prostą, ale warto się nad tym zagadnieniem pochylić, ponieważ tylko przez budowanie i obserwowanie prostych modeli w działaniu, będziemy w stanie lepiej rozumieć i stopniowo ulepszać modele bardziej złożone.
Metoda, którą tutaj pokazałem ma wiele ograniczeń, ale na pewno istnieje w niej pewien potencjał, który jest warty poznania i dalszych rozważań, do których niniejsza praca miała zachęcić.